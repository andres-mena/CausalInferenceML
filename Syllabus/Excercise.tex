\documentclass[12pt]{article}
\usepackage{geometry}
\geometry{margin=1in}
\usepackage{enumitem}
\usepackage{hyperref}
\usepackage{amsmath}
\usepackage{booktabs}

\title{Causal Inference Using Machine Learning\\
Master in Economics\\
Universidad Nacional de Tucumán\\
Spring 2024}
\author{Instructor: Andres Mena (asmena@face.unt.edu.ar)}
\date{}

\begin{document}
\maketitle

\section*{Evaluation Criteria}

The final grade for this course will be determined based on four assignments:

\begin{itemize}
    \item Replication Exercise: \textbf{20\%}
    \item Midterm Presentation: \textbf{10\%}
    \item Final Presentation: \textbf{20\%}
    \item Research Proposal: \textbf{50\%}
\end{itemize}

\section*{Assignment Descriptions}

\subsection*{1. Replication Exercise (20\%)}
The replication exercise involves selecting a paper from a top 5 journal in economics that addresses a causal question of interest. The top five journals are:

\begin{itemize}
    \item \textit{American Economic Review}
    \item \textit{Quarterly Journal of Economics}
    \item \textit{Journal of Political Economy}
    \item \textit{Econometrica}
    \item \textit{Review of Economic Studies}
\end{itemize}

Your task is to replicate the main table or result of interest using the actual code and data provided by the paper. A 2-page document should be submitted, which includes:

\begin{itemize}
    \item A brief summary of the paper
    \item The research question of the paper
    \item Identification strategy used by the authors
    \item Data description
    \item Replicated tables or plots produced by you
\end{itemize}

\textbf{Deadline: November 20th, 2024}

\subsection*{2. Midterm Presentation (10\%)}
The midterm presentation is a 20-minute presentation where you should prepare slides describing:

\begin{itemize}
    \item The causal question of interest, clearly identifying the treatment and outcome
    \item An ideal experiment that, without restrictions on budget, data, or ethical considerations, would allow you to answer the question
    \item Potential data sources to empirically address the question
    \item Identification strategy, including methodology and assumptions
    \item Challenges for implementation
\end{itemize}

\textbf{Presentation Date: November 25th, 2024}

\subsection*{3. Final Presentation (20\%)}
The final presentation is an expanded version of your project, including preliminary results, tables, and plots. This will be a 30-minute presentation where you are expected to:

\begin{itemize}
    \item Present updated research questions and objectives
    \item Show preliminary empirical results
    \item Discuss tables and plots that illustrate your findings
    \item Address any challenges faced and plans to overcome them
\end{itemize}

\textbf{Presentation Date: December 19th, 2024}

\subsection*{4. Research Proposal (50\%)}
The research proposal is a 5-page document structured like an academic paper, which should include:

\begin{itemize}
    \item \textbf{Abstract}: A brief summary of the research question and findings
    \item \textbf{Introduction}: Context and motivation for the study
    \item \textbf{Setup}: Description of the theoretical framework or model
    \item \textbf{Identification Strategy}: Detailed methodology and assumptions used to identify causal effects
    \item \textbf{Results}: Presentation of empirical findings
    \item \textbf{Conclusion}: Summary of results and implications
\end{itemize}

If required, additional plots, tables, or information should be included in the appendix.

\textbf{Deadline: February 10th, 2025}

\end{document}
